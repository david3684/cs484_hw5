%%%%%%%%%%%%%%%%%%%%%%%%%%%%%%%%%%%%%%%%%%%%%%%%%%%%%%%%%%%%%%%%%%%%%%%%%%%%%%%%%%%%%%%%%%%%%%%%
%
% CS484 Written Question Template
%
% Acknowledgements:
% The original code is written by Prof. James Tompkin (james_tompkin@brown.edu).
% The second version is revised by Prof. Min H. Kim (minhkim@kaist.ac.kr).
%
% This is a LaTeX document. LaTeX is a markup language for producing 
% documents. Your task is to fill out this document, then to compile 
% it into a PDF document. 
%
% 
% TO COMPILE:
% > pdflatex thisfile.tex
%
% If you do not have LaTeX and need a LaTeX distribution:
% - Personal laptops (all common OS): www.latex-project.org/get/
% - We recommend latex compiler miktex (https://miktex.org/) for windows,
%   macTex (http://www.tug.org/mactex/) for macOS users.
%   And TeXstudio(http://www.texstudio.org/) for latex editor.
%   You should install both compiler and editor for editing latex.
%   The another option is Overleaf (https://www.overleaf.com/) which is 
%   an online latex editor.
%
% If you need help with LaTeX, please come to office hours. 
% Or, there is plenty of help online:
% https://en.wikibooks.org/wiki/LaTeX
%
% Good luck!
% Min and the CS484 staff
%
%%%%%%%%%%%%%%%%%%%%%%%%%%%%%%%%%%%%%%%%%%%%%%%%%%%%%%%%%%%%%%%%%%%%%%%%%%%%%%%%%%%%%%%%%%%%%%%%
%
% How to include two graphics on the same line:
% 
% \includegraphics[\width=0.49\linewidth]{yourgraphic1.png}
% \includegraphics[\width=0.49\linewidth]{yourgraphic2.png}
%
% How to include equations:
%
% \begin{equation}
% y = mx+c
% \end{equation}
% 
%%%%%%%%%%%%%%%%%%%%%%%%%%%%%%%%%%%%%%%%%%%%%%%%%%%%%%%%%%%%%%%%%%%%%%%%%%%%%%%%%%%%%%%%%%%%%%%%

\documentclass[11pt]{article}

\usepackage[english]{babel}
\usepackage[utf8]{inputenc}
\usepackage[colorlinks = true,
            linkcolor = blue,
            urlcolor  = blue]{hyperref}
\usepackage[a4paper,margin=1.5in]{geometry}
\usepackage{stackengine,graphicx}
\usepackage{fancyhdr}
\setlength{\headheight}{15pt}
\usepackage{microtype}
\usepackage{times}
\usepackage{booktabs}
\usepackage{listings}
\usepackage{xcolor}
\lstdefinestyle{codestyle}{
	frame=single,
	basicstyle=\ttfamily\footnotesize,
	keywordstyle=\bfseries\color{magenta},
	commentstyle=\itshape\color{gray},
	stringstyle=\color{orange},
	numberstyle=\sffamily\scriptsize\color{gray},
	showspaces=false,
	showstringspaces=false,
	showtabs=false,
	tabsize=4,
	breakatwhitespace=false,
	breaklines=true,
	keepspaces=true,
	captionpos=b,
	numbers=left,
	numbersep=5pt}
\lstset{style=codestyle}

\frenchspacing
\setlength{\parindent}{0cm} % Default is 15pt.
\setlength{\parskip}{0.3cm plus1mm minus1mm}

\pagestyle{fancy}
\fancyhf{}
\lhead{Homework Writeup}
\rhead{CS484}
\rfoot{\thepage}

\date{}

\title{\vspace{-1cm}Homework 5 Writeup}


\begin{document}
\maketitle
\vspace{-3cm}
\thispagestyle{fancy}

\section*{Code Implementation and Details}
\subsection*{Feature Extraction}
\begin{lstlisting}[language=python]
def feature_extraction(img, feature):
    """
    This function computes defined feature (HoG, SIFT) descriptors of the target image.

    :param img: a height x width x channels matrix,
    :param feature: name of image feature representation.

    :return: a number of grid points x feature_size matrix.
    """

    if feature == 'HoG':
        # HoG parameters
        win_size = (32, 32)
        block_size = (32, 32)
        block_stride = (16, 16)
        cell_size = (16, 16)
        nbins = 9
        deriv_aperture = 1
        win_sigma = 4
        histogram_norm_type = 0
        l2_hys_threshold = 2.0000000000000001e-01
        gamma_correction = 0
        nlevels = 64
        
        hog = cv2.HOGDescriptor(win_size, block_size, block_stride, cell_size, nbins, deriv_aperture, win_sigma, histogram_norm_type, l2_hys_threshold, gamma_correction, nlevels)
        hog_features = hog.compute(img)
        
        return hog_features.reshape(-1, 36)
    elif feature == 'SIFT':
        sift = cv2.SIFT_create()
        step_size = 20
        keypoints = [cv2.KeyPoint(x, y, step_size) for y in range(0, img.shape[0], step_size) for x in range(0, img.shape[1], step_size)]
        keypoints, descriptors = sift.compute(img, keypoints)
        
        
        if descriptors is None:
            return np.zeros((0, 128), dtype=np.float32)
        return descriptors
\end{lstlisting}

\begin{itemize}
    \item \textbf{HoG Descriptor} For HoG Descriptor, I simply used cv2.HoGDescriptor function to obtain HoG feature points. Then I reshaped the output into -1*36 size, which automatically makes the row number.
    \item \textbf{SIFT Descriptor} For SIFT Descriptor, first I devided the image into 20*20 size grids, and for each grid made keypoints with cv2.Keypoint function. Then for each keypoints I computed the SIFT descriptor. If there are no descriptors found, the code gives zero array for output.
\end{itemize}
\subsection*{Principal Component Analysis}
\begin{lstlisting}[language=python]
def get_features_from_pca(feat_num, feature):

    vocab = np.load(f'vocab_{feature}.npy')
    
    vocab_mean = np.mean(vocab, axis=0)
    vocab_centered = vocab - vocab_mean

    covariance_matrix = np.cov(vocab_centered, rowvar=False)

    eigen_values, eigen_vectors = np.linalg.eigh(covariance_matrix)

    idx = np.argsort(eigen_values)[::-1]
    principal_components = eigen_vectors[:, idx[:feat_num]]

    reduced_vocab = np.dot(vocab_centered, principal_components)

    return reduced_vocab
\end{lstlisting}
\begin{itemize}
    \item First, from the loaded vocabulary I centered it by extracting the mean.
    \item Then, with the centered vocabulary I calculated the covariance matrix, and performed eigenvalue decompostition to obtain eigenvalues and vectors.
    \item I sorted the eigenvalues from largest one to the smallest one, and then chose the eigenvectors corresponding to largest feat\_num indices. 
    \item Finally, I obtained the reduced vocabulary by making a dot product with the centered vocabulary and selected eigenvectors.
\end{itemize}
\subsection*{Bag of Words}
\begin{lstlisting}[language=python]
def get_bags_of_words(image_paths, feature):
    vocab = np.load(f'vocab_{feature}.npy')

    vocab_size = vocab.shape[0]

    bags_of_words = np.zeros((len(image_paths), vocab_size))

    for i, path in enumerate(image_paths):
        img = cv2.imread(path)
        features = feature_extraction(img, feature)

        distances = pdist(features, vocab)
        closest_vocab_indices = np.argmin(distances, axis=1)

        for idx in closest_vocab_indices:
            bags_of_words[i, idx] += 1

        bags_of_words[i, :] /= linalg.norm(bags_of_words[i, :])

    return bags_of_words
\end{lstlisting}
\section*{In the beginning...}

Lorem ipsum dolor sit amet, consectetur adipisicing elit, sed do eiusmod tempor incididunt ut labore et dolore magna aliqua. Ut enim ad minim veniam, quis nostrud exercitation ullamco laboris nisi ut aliquip ex ea commodo consequat. Duis aute irure dolor in reprehenderit in voluptate velit esse cillum dolore eu fugiat nulla pariatur. Excepteur sint occaecat cupidatat non proident, sunt in culpa qui officia deserunt mollit anim id est laborum. See Equation~\ref{eq:one}.

\begin{equation}
a = b + c
\label{eq:one}
\end{equation}

\section*{Interesting Implementation Detail}

Lorem ipsum dolor sit amet, consectetur adipisicing elit, sed do eiusmod tempor incididunt ut labore et dolore magna aliqua. Ut enim ad minim veniam, quis nostrud exercitation ullamco laboris nisi ut aliquip ex ea commodo consequat. Duis aute irure dolor in reprehenderit in voluptate velit esse cillum dolore eu fugiat nulla pariatur. Excepteur sint occaecat cupidatat non proident, sunt in culpa qui officia deserunt mollit anim id est laborum.

My code snippet highlights an interesting point.
\begin{lstlisting}[language=python]
import numpy as np
one = 1
two = one + one
if two == 2:
	# This is comment
    print('This computer is not broken.')
else:
    print('This computer is broken.')
\end{lstlisting}

\section*{A Result}

\begin{enumerate}
    \item Result 1 was a total failure, because...
    \item Result 2 (Figure~\ref{fig:result1}, left) was surprising, because...
    \item Result 3 (Figure~\ref{fig:result1}, right) blew my socks off, because...
\end{enumerate}

\begin{figure}[h]
    \centering
    \includegraphics[width=5cm]{placeholder.jpg}
    \includegraphics[width=5cm]{placeholder.jpg}
    \caption{\emph{Left:} My result was spectacular. \emph{Right:} Curious.}
    \label{fig:result1}
\end{figure}

My results are summarized in Table~\ref{tab:table1}.

\begin{table}[h]
    \centering
    \begin{tabular}{lr}
        \toprule
        Condition & Time (seconds) \\
        \midrule
        Test 1 & 1 \\
        Test 2 & 1000 \\
        \bottomrule
    \end{tabular}
    \caption{Stunning revelation about the efficiency of my code.}
    \label{tab:table1}
\end{table}

\end{document}